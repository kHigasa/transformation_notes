\part{数学} % (fold)
\label{prt:数学}
\chapter*{算数} % (fold)
\label{cha:算数}
\begin{align}
    (\cot x)' &= -\frac{1}{\sin^2x} \\
    \text{cf. }(\tan x)' &= \frac{1}{\cos^2x}\notag
\end{align}
note: 適当な微分. 
\begin{equation}
\label{不定積分1}
    \int\frac{1}{\sqrt{x^2+A}}\dd{x} = \log(x+\sqrt{x^2+A})+\text{const. }(A>0)
\end{equation}
note: 被積分函数分母の双曲線を見て$x=\sqrt{A}\sinh t$と置換, なんて気怠いことはしない. 
\begin{align}
\label{逆双曲線関数}
    \mathrm{arcsinh}\ x &= \log(x+\sqrt{x^2+1}) \\
    \text{cf. }\mathrm{arccosh}\ x &= \log(x\pm\sqrt{x^2-1})\notag
\end{align}
note: 逆双曲線函数が現れる奇妙さを覚えておくのもいい. 
\begin{align}
\label{不定積分2}
    \int\frac{1}{\sqrt{a^2-x^2}}\dd{x} &= \arcsin\frac{x}{a}+\text{const.} \\
    \text{cf. }\int\frac{1}{x^2+a^2}\dd{x} &= \frac{1}{a}\arctan\frac{x}{a}+\text{const.}\notag
\end{align}
note: 三角関数についても全く同様. 
\begin{align}
    \int\sqrt{x^2+A}\dd{x} &= \frac{1}{2}\qty(x\sqrt{x^2+A}+A\log(x+\sqrt{x^2+A}))+\text{const. }(A>0) \\
    \int\sqrt{a^2-x^2}\dd{x} &= \frac{1}{2}\qty(x\sqrt{a^2-x^2}+a^2\arcsin\frac{x}{a})+\text{const.}
\end{align}
note: 不定積分(\ref{不定積分1})と(\ref{不定積分2})のまとめ的なものとして. 

$a,b,c,d,e,f\in\mathbb{Z}$, $p,q,r,\alpha,\beta,\gamma\in\mathbb{Q}$として, 
\begin{align}
    \frac{px+q}{(ax+b)(cx+d)} &= \frac{\alpha}{ax+b}+\frac{\beta}{cx+d} \\
    \frac{px+q}{(ax+b)^2} &= \frac{\alpha}{ax+b}+\frac{\beta}{(ax+b)^2} \\
    \frac{px^2+qx+r}{(ax+b)(cx+d)(ex+f)} &= \frac{\alpha}{ax+b}+\frac{\beta}{cx+d}+\frac{\gamma}{ex+f} \\
    \frac{px^2+qx+r}{(ax+b)^2(cx+d)} &= \frac{\alpha}{ax+b}+\frac{\beta}{(ax+b)^2}+\frac{\gamma}{cx+d} \\
    \frac{px^2+qx+r}{(ax+b)(cx^2+dx+e)} &= \frac{\alpha}{ax+b}+\frac{\beta x+\gamma}{cx^2+dx+e}
\end{align}
note: 部分分数分解. 

場合の数での用途を想定し, $n,k\in\mathbb{N}$として, 
\begin{align}
    \text{重複Combination: }n\text{H}k &= \frac{(n+k-1)!}{k!(n-1)!} \\
    \text{cf. }\text{Combination: }n\text{C}k &= \frac{n!}{k!(n-k)!}\notag \\
    \text{cf. }\text{Permutation: }n\text{P}k &= \frac{n!}{(n-k)!}\notag
\end{align}
note: 代数表示. 
% chapter 算数 (end)

\chapter{線形代数} % (fold)
\label{cha:線形代数}
\section{量} % (fold)
\label{sec:量}
\begin{dfn}
    aaa
    \begin{align}
        E = mc \pushQED{\qed}
    \end{align}
\end{dfn}
\begin{theo}
    aaa
    \begin{align}
        E = mc \pushQED{\qed}
    \end{align}
\end{theo}
\begin{prop}
    aaa
\end{prop}
\begin{lemma}
    aaa
\end{lemma}
\begin{cor}
    aaa
\end{cor}
% section 量 (end)
% chapter 線形代数 (end)

\chapter{微分方程式} % (fold)
\label{cha:微分方程式}
\section{常微分方程式} % (fold)
\label{sec:常微分方程式}
\begin{theo}[定数係数2階線形微分方程式]
    $a$, $b$を実定数とし, 2階斉次微分方程式$y''+ay'+by=0$を考える.特性方程式$\lambda^2+a\lambda+b=0$の2根を$\lambda_{1}$, $\lambda_{2}$とする. このとき, 一般解$y(x)$は, $C_{1}$, $C_{2}$を任意定数として, 
    \begin{equation}
        y(x)= \begin{cases}
                C_{1}e^{\lambda_{1} t}+C_{2}e^{\lambda_{2} t} &\text{when }\lambda_{1}\neq\lambda_{2}, (\lambda_{1}, \lambda_{2}\in\mathbb{R}) \\
                C_{1}e^{\lambda_{1} t}+C_{2}te^{\lambda_{1} t} &\text{when }\lambda_{1}=\lambda_{2} \\
                e^{\alpha t}(C_{1}cos{\beta t}+C_{2}sin{\beta t}) &\text{when }\lambda_{2}=\alpha+{i\beta}, \lambda_{2}=\alpha-{i\beta}, (\alpha, \beta\in\mathbb{R}, \beta>0)
        \end{cases}
    \end{equation}
\end{theo}
で与えられる.
% section 常微分方程式 (end)
% chapter 微分方程式 (end)
% part 数学 (end)