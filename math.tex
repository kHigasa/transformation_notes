\part{数学} % (fold)
\label{prt:数学}
\chapter*{算数} % (fold)
\label{cha:算数}

% chapter 算数 (end)

\chapter{線形代数} % (fold)
\label{cha:線形代数}
\section{量} % (fold)
\label{sec:量}
\begin{dfn}
    aaa
    \begin{align}
        E = mc \pushQED{\qed}
    \end{align}
\end{dfn}
\begin{theo}
    aaa
    \begin{align}
        E = mc \pushQED{\qed}
    \end{align}
\end{theo}
\begin{prop}
    aaa
\end{prop}
\begin{lemma}
    aaa
\end{lemma}
\begin{cor}
    aaa
\end{cor}
% section 量 (end)
% chapter 線形代数 (end)

\chapter{微分方程式} % (fold)
\label{cha:微分方程式}
\section{常微分方程式} % (fold)
\label{sec:常微分方程式}
\begin{theo}[定数係数2階線形微分方程式]
    $a$, $b$を実定数とし, 2階斉次微分方程式$y''+ay'+by=0$を考える.特性方程式$\lambda^2+a\lambda+b=0$の2根を$\lambda_{1}$, $\lambda_{2}$とする. このとき, 一般解$y(x)$は, $C_{1}$, $C_{2}$を任意定数として, 
    \begin{equation}
        y(x)= \begin{cases}
                C_{1}e^{\lambda_{1} t}+C_{2}e^{\lambda_{2} t} &\text{when }\lambda_{1}\neq\lambda_{2}, (\lambda_{1}, \lambda_{2}\in\mathbb{R}) \\
                C_{1}e^{\lambda_{1} t}+C_{2}te^{\lambda_{1} t} &\text{when }\lambda_{1}=\lambda_{2} \\
                e^{\alpha t}(C_{1}cos{\beta t}+C_{2}sin{\beta t}) &\text{when }\lambda_{2}=\alpha+{i\beta}, \lambda_{2}=\alpha-{i\beta}, (\alpha, \beta\in\mathbb{R}, \beta>0)
        \end{cases}
    \end{equation}
\end{theo}
で与えられる.
% section 常微分方程式 (end)
% chapter 微分方程式 (end)
% part 数学 (end)