\documentclass[a4paper, 12pt]{jsreport}
\input preemble.tex
\input title.tex

% prevent hyphenation
\hyphenpenalty=10000\relax
\exhyphenpenalty=10000\relax
\sloppy

\begin{document}
\maketitle

\input preface.tex

\tableofcontents

\part{数学定義の章}
\chapter{線形代数}
線形代数に関する定義を掲載しています.
\section{正則行列}
$n$次正方行列$A$に対して$AX=XA=E_n$を満たす$n$次正方行列(逆行列という)$X$($A^{-1}$)が存在するとき, $A$は正則行列であるという.

\section{随伴行列}
($m$, $n$)型行列に対して, 共軛転置${}^{t}\overline{A}$を随伴行列という.

\section{Hermite行列・対称行列}
$A^*=A$ $\Leftrightarrow$ $n$次正方行列$A$がHermite行列.\\
また, 対称行列とは実Hermite行列のこと.

\section{Unitary行列・直交行列}
$A^*A=E_n$ $\Leftrightarrow$ $n$次正方行列$A$がUnitary行列.\\
また, 直交行列とは実Unitary行列のこと.

\part{数学公式の章}
\chapter{三角関数}
三角関数に関する公式・式変形等を掲載しています.
\section{3倍角の公式}
\begin{equation}
    \begin{array}{l}
        \sin3\theta=3\sin\theta-4\sin^3\theta\\
        \cos3\theta=4\cos^3\theta-3\cos\theta
    \end{array}
\end{equation}
\subsection{導出の手続き}
hoge
\subsubsection{注意}
hoge

\section{積和の公式}
\begin{equation}
    \begin{array}{l}
        \sin\alpha\cos\beta=\frac{\sin(\alpha+\beta)+\sin(\alpha-\beta)}{2}\\
        \cos\alpha\sin\beta=\frac{\sin(\alpha+\beta)-\sin(\alpha-\beta)}{2}\\
        \sin\alpha\sin\beta=-\frac{\cos(\alpha+\beta)-\cos(\alpha-\beta)}{2}\\
        \cos\alpha\cos\beta=-\frac{\cos(\alpha+\beta)+\cos(\alpha-\beta)}{2}
    \end{array}
\end{equation}
\subsection{導出の手続き}
加法定理の加減から導出する.
\subsubsection{注意}
hoge

\section{和積の公式}
\begin{equation}
    \begin{array}{l}
        \sin A+\sin B=2\sin\frac{A+B}{2}\cos\frac{A-B}{2}\\
        \sin A-\sin B=2\cos\frac{A+B}{2}\sin\frac{A-B}{2}\\
        \cos A+\cos B=2\cos\frac{A+B}{2}\cos\frac{A-B}{2}\\
        \cos A-\cos B=-2\sin\frac{A+B}{2}\sin\frac{A-B}{2}
    \end{array}
\end{equation}
\subsection{導出の手続き}
積和の公式から導出する.
\subsubsection{注意}
$\alpha+\beta$と置き換える.

\chapter{積分}
\section{6分の1公式}

\section{12分の1公式}

\chapter{線形代数}
\section{Schwartzの不等式・三角不等式}
\begin{equation}
    \begin{array}{l}
        |(\bm{a}, \bm{b})|\le\|\bm{a}\|\cdot\|\bm{b}\|\\
        \|\bm{a}+\bm{b}\|\le\|\bm{a}\|+\|\bm{b}\|
    \end{array}
\end{equation}

\chapter{複素解析}

\part{数学の簡便な結論を纏めた章}
\chapter{微分方程式}
\section{常微分方程式}
\subsection{定数係数2階線形微分方程式}
$a$, $b$を実定数とし, 2階斉次微分方程式$y''+ay'+by=0$を考える.特性方程式$\lambda^2+a\lambda+b=0$の2根を$\lambda_{1}$, $\lambda_{2}$とする. このとき, 一般解$y(x)$は, $C_{1}$, $C_{2}$を任意定数として, 
\begin{equation}
    y(x)= \left \{
        \begin{array}{l}
            C_{1}e^{\lambda_{1} t}+C_{2}e^{\lambda_{2} t}\\
            \ \ \mathrm{when} \ \lambda_{1}\neq\lambda_{2}, (\lambda_{1}, \lambda_{2}\in\mathbb{R}) \\
            C_{1}e^{\lambda_{1} t}+C_{2}te^{\lambda_{1} t}\\
            \ \ \mathrm{when} \ \lambda_{1}=\lambda_{2} \\
            e^{\alpha t}(C_{1}cos{\beta t}+C_{2}sin{\beta t}) \\
            \ \ \mathrm{when} \ \lambda_{2}=\alpha+{i\beta}, \lambda_{2}=\alpha-{i\beta}, (\alpha, \beta\in\mathbb{R}, \beta>0)
        \end{array}
    \right.
\end{equation}
で与えられる.

\part{物理学の章}
\chapter{物理量}
\section{慣性モーメント}
\section{波長}

\section{波数}

\chapter{近似式}

\end{document}
