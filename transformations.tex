\documentclass[]{jreport}
\usepackage{amsmath, amssymb}
\usepackage{type1cm}
\begin{document}

\part{はじめに}
\chapter{概要}

まず,この文書では私がよく使うくせに自己の記憶媒体に中々収まりきらない式変形等をまとめたものである.

\part{数学公式の章}
\chapter{三角関数}

三角関数に関する公式・式変形を掲載しています.

\section{3倍角の公式}
\section{和積の公式}

$x$ $\ll$ 1のとき,
\begin{equation}
    \frac{\pi}{2} =
    \left( \int_{0}^{\infty} \frac{\sin x}{\sqrt{x}} dx \right)^2 =
    \sum_{k=0}^{\infty} \frac{(2k)!}{2^{2k}(k!)^2} \frac{1}{2k+1} =
    \prod_{k=2}^{\infty} \frac{4k^2}{4k^2 - 1}
\end{equation}

\subsection{覚え方}

ちょうちょちょうちょ

\subsubsection{注意}

実際によくつかうものを掲載しています.

\chapter{積分}

\section{6分の1公式}
\section{12分の1公式}

\part{数学の簡便な結論を纏めた章}
\chapter{微分方程式}
\section{常微分方程式}
\subsection{定数係数2階線形微分方程式}

$a$, $b$を実定数とし, 2階斉次微分方程式$y''+ay'+by=0$を考える.特性方程式$\lambda^2+a\lambda+b=0$の2根を$\lambda_{1}$, $\lambda_{2}$とする. このとき, 一般解$y(x)$は, $C_{1}$, $C_{2}$を任意定数として, \\
\begin{equation}
    y(x)= \left \{
        \begin{array}{l}
            C_{1}e^{\lambda_{1} t}+C_{2}e^{\lambda_{2} t}\\
            \ \ when \ \lambda_{1}\neq\lambda_{2}, (\lambda_{1}, \lambda_{2}\in\mathbb{R}) \\
            C_{1}e^{\lambda_{1} t}+C_{2}te^{\lambda_{1} t}\\
            \ \ when \ \lambda_{1}=\lambda_{2} \\
            e^{\alpha t}(C_{1}cos{\beta t}+C_{2}sin{\beta t}) \\
            \ \ when \ \lambda_{1}=\alpha+{i\beta}, \lambda_{2}=\alpha-{i\beta}, (\alpha, \beta\in\mathbb{R}, \beta>0)
        \end{array}
    \right.
\end{equation}
で与えられる.

\part{物理学の章}
\chapter{物理量}

test

\section{慣性モーメント}

test

\chapter{近似式}

test

\end{document}
