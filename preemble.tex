\usepackage{amsmath, amssymb, amsthm}
\usepackage{bm}
\usepackage{physics}
\usepackage{siunitx}
\usepackage{url}
\usepackage[dvipdfmx]{graphicx, color}
\usepackage[dvipdfmx]{hyperref}
\usepackage{pxjahyper}
\hypersetup{
    setpagesize=false,
    bookmarksnumbered=true,
    bookmarksopen=true,
    colorlinks=true,
    anchorcolor=blue,
    linkcolor=black,
    citecolor=blue,
    urlcolor=cyan,
}
\usepackage{tikz}
\usetikzlibrary{decorations, decorations.pathreplacing}
\usepackage{circuitikz}
\usepackage{tcolorbox}
\tcbuselibrary{theorems, skins}
\usepackage{ascmac}
\usepackage[numbers, sort]{natbib}

\newtheoremstyle{mystyle}%   % スタイル名
    {}%                      % 上部スペース
    {}%                      % 下部スペース
    {\normalfont}%           % 本文フォント
    {}%                      % インデント量
    {\bf}%                   % 見出しフォント
    {}%                      % 見出し後の句読点, '.'
    {\newline}%              % 見出し後のスペース, ' ' or \newline
    {\thmname{#1}\thmnumber{#2}\thmnote{(#3)}}% % 見出しの書式 (can be left empty, meaning `normal')
\theoremstyle{mystyle}
\newtheorem{theo}{Th.}[section]
\newtheorem{dfn}[theo]{Def.}
\newtheorem{prop}[theo]{Prop.}
\newtheorem{lemma}[theo]{Lem.}
\newtheorem{cor}[theo]{Cor.}